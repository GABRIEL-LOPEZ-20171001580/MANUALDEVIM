\title{Manual de uso del editor de texto \texttt{Vim}}
\author{Gabriel López}
\date{\today}

\documentclass[10pt]{article}
\usepackage[spanish]{babel}
\usepackage{graphicx}
\usepackage[letterpaper]{geometry}


\begin{document}
\maketitle

\section{\texttt{Vim}, un editor de texto minimalista.}
\texttt{Vim} es un editor de texto creado por Brian Moolenaar. Es una versión mejorada del editor de texto \texttt{vi} creado por el programador Bill Joy. 
El editor de texto \texttt{Vim} es un editor de texto pensando para ser un programa de edición de texto minimalista y extensible, lo que significa que con ayuda de programas auxiliares hechos por la comunidad conocidos como \textit{plugins}, se puede utilizar el editor \texttt{Vim} como un entorno de programación (conocido como \textit{IDE}) o un entorno minimalista de edición de documentos hechos en \LaTeX. \\ 
Si bien, aprender a usar \texttt{Vim} es un proyecto con un curva de aprendizaje un poco pronunciada, este editor ofrece ciertas ventajas por sobre otros editores, sobre todo a nivel de personalización, configuración y uso de recursos, lo que lo hace ídoneo en \textit{hardware} de capacidad limitada. 
Además de esto, con la correcta configuración y el correcto uso de \textit{plugins}, la escritura (sobre todo de documentos de tipo científico) puede ser más fluida a un nivel muy similar a la toma de notas manuscritas. 
\section{Entornos en \texttt{Vim}}
\texttt{Vim} es un editor de texto poco ortodoxo comparado con el resto de los editores de texto que están disponibles para uso público, para empezar, \texttt{Vim} usa un esquema de entornos \textit{entornos}, lo que quiere decir, que para las acciones que usualmente se ven relegadas al uso del ratón en otros editores, en \texttt{Vim} están relegadas a un \textit{entorno}. Los entornos de uso básico son los siguientes:
\begin{itemize}
	\item \textit{NORMAL}: Este entorno es el entorno básico de \texttt{Vim}. En este entorno se utiliza para navegar entre líneas de texto, borrar contenido y introducir comandos dentro del editor.
	      La utilidad del modo \textit{NORMAL} radica en el uso de comandos para usar la terminal y el editor al mismo tiempo. 
	\item \textit{INSERTAR}: Este entorno se utiliza para la inserción de texto dentro de líneas en el editor. Para entrar a este entonro basta con introducir la tecla \texttt{i} mediante el teclado.
Una vez que la tecla se ha introducido, el editor colocará el texto \texttt{INSERTAR} o \texttt{INSERT} en la parte inferior de la pantalla, denoatando que el modo de inserción está activado. 
	\item \textit{VISUAL}: Este entorno se encarga de función de copiar y pegar contenido de las lineas. Para entrar a este modo, se debe introducir \texttt{v} con el teclado.
Una vez que el modo \textit{VISUAL} está activado debe aparecer en pantalla la leyenda \texttt{VISUAL} en la esquina inferior izquierda. 
\end{itemize}
Es importante destacar que para salir de un entorno para entrar de nuevo al entorno \textit{NORMAL} se introduce la tecla \texttt{Esc} con el teclado. Para ilustrar los entornos de \texttt{Vim} colocamos la siguiente imagen. En dicha imagen es notorio que debido al idioma con el que se tiene el ordenador, aparece la leyenda \texttt{INSERTAR} en la parte inferior, denotando que el modo de inserción de texto está activo. 
\begin{figure}[h]
	\centering
	\includegraphics[scale=0.6]{./img/Vim_INS.png}
	\caption{\texttt{Vim} en el modo \textit{INSERTAR}}
\end{figure}
\newline
A pesar de todo esto, existen muchos más entornos dentro del editor, sin embargo, consideramos que estos tres son los más útiles para un usuario nuevo de \texttt{Vim} debido a que facilitan la escritura de documentos y permiten la introducción de comandos dentro del mismo. 
\newpage
\section{¿Cómo salir de \texttt{Vim}?}
Es normal que el usuario novato de GNU-Linux o cualquier sistema operativo derivado de UNIX. De modo que el sistema operativo a veces deja el usuario dentro de \texttt{Vim} sin ningún aviso de cómo salir del editor de texto. Por esta razón en esta sección se explicará como salir de \texttt{Vim}. \\
Para empezar, la imagen inicial que el usuario tiene al entrar a \texttt{Vim} es la siguiente: \newline
\begin{figure}[h]
	\centering
	\includegraphics[scale=0.4]{./img/poisson_mu1.png}
	\caption{Primera impresión de \texttt{Vim}}
\end{figure}
\newline
Podría parecer que \texttt{Vim} es complicado, pero en realidad, salir del editor es relativamente sencillo, la secuencia de pasos para salir de \texttt{Vim} es la siguiente:
\begin{itemize}
	\item Primero, si usted desea guardar el contenido del archivo, es conveniente guardar el contenido del archivo primero, esto se logra mediante el comando \texttt{:w} dentro del entorno \textit{NORMAL}.
	\item Ahora que el archivo está guardado, es posible salir del editor sin problemas, por lo que se procede a colocar el comando \texttt{:q} dentro del mismo entorno \textit{NORMAL}.
\end{itemize}
Si bien es posible que queramos guardar el contenido de nuestro archivo de texto, también tenemos la opción de salir sin guardar (no recomendada a menos que se desee salir de \texttt{Vim} si se entró por error), en este caso, basta con introducir el comando \texttt{:q!} en el modo \textit{NORMAL} del editor.
\section{¿Existe alguna forma de aprender \texttt{Vim} mediante ejemplos?}
Si bien \texttt{Vim} es un editor de texto que puede parecer poco ortodoxo al principio, existen formas para aprender su uso de una forma interactiva mediante ejemplos. En el caso de \texttt{Vim} existe el complemento denominado \texttt{vimtutor}, que es un complemento de consola que suele estar instalado en cualquier máquina UNIX. En el caso de querer acceder a dicho complemento, basta con introducir el comando \texttt{vimtutor} dentro de la terminal de GNU-Linux o su equivalente en UNIX. 
\section{Navegación en \texttt{Vim}}
\texttt{Vim} es un editor de texto que tiene un esquema de navegación diferente al resto de editores de texto que podemos encontrar en el mercado. Esto es debido a cuestiones de convención. 
Si bien, las teclas direccionales del teclado funcionan para navegar entre texto dentro del editor, su uso no es recomendado por la mayoría de guias de uso de \texttt{Vim}. \\
Esto se entiende mejor con los ejemplos proporcionados por el complemento de consola llamado \texttt{vimtutor}. Para nuestros efectos, se busca introducir al usuario a los conceptos básicos de los que \texttt{Vim} para que sea capaz de redactar documentos simples dentro del editor. Dicho esto, podemos mostrar el esquema básico de navegación del que hace uso \texttt{Vim}.
\begin{figure}[h]
\begin{center}
	\includegraphics[scale=0.4]{./img/vim_nav.png}
\end{center}
\caption{Esquema de navegación básico dentro de \texttt{Vim}}
\label{fig:}
\end{figure}

\end{document}

