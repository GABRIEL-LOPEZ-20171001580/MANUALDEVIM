\title{Manual de uso del editor de texto \texttt{Vim}}
\author{Gabriel López}
\date{\today}

\documentclass[10pt]{article}
\usepackage[spanish]{babel}
\usepackage{graphicx}
\usepackage[letterpaper]{geometry}


\begin{document}
\maketitle

\section{\texttt{Vim}, un editor de texto minimalista.}
\texttt{Vim} es un editor de texto creado por Brian Moolenaar. Es una versión mejorada del editor de texto \texttt{vi} creado por el programador Bill Joy. 
El editor de texto \texttt{Vim} es un editor de texto pensando para ser un programa de edición de texto minimalista y extensible, lo que significa que con ayuda de programas auxiliares hechos por la comunidad conocidos como \textit{plugins}, se puede utilizar el editor \texttt{Vim} como un entorno de programación (conocido como \textit{IDE}) o un entorno minimalista de edición de documentos hechos en \LaTeX. \\ 
Si bien, aprender a usar \texttt{Vim} es un proyecto con un curva de aprendizaje un poco pronunciada, este editor ofrece ciertas ventajas por sobre otros editores, sobre todo a nivel de personalización, configuración y uso de recursos, lo que lo hace ídoneo en \textit{hardware} de capacidad limitada. 
Además de esto, con la correcta configuración y el correcto uso de \textit{plugins}, la escritura (sobre todo de documentos de tipo científico) puede ser más fluida a un nivel muy similar a la toma de notas manuscritas. 
\section{Entornos en \texttt{Vim}}
\texttt{Vim} es un editor de texto poco ortodoxo comparado con el resto de los editores de texto que están disponibles para uso público, para empezar, \texttt{Vim} usa un esquema de entornos \textit{entornos}, lo que quiere decir, que para las acciones que usualmente se ven relegadas al uso del ratón en otros editores, en \texttt{Vim} están relegadas a un \textit{entorno}. Los entornos de uso básico son los siguientes:
\begin{itemize}
	\item \textit{NORMAL}: Este entorno es el entorno básico de \texttt{Vim}. En este entorno se utiliza para navegar entre líneas de texto, borrar contenido y introducir comandos dentro del editor.
	      La utilidad del modo \textit{NORMAL} radica en el uso de comandos para usar la terminal y el editor al mismo tiempo. 
	\item \textit{INSERTAR}: Este entorno se utiliza para la inserción de texto dentro de líneas en el editor. Para entrar a este entonro basta con introducir la tecla \texttt{i} mediante el teclado.
Una vez que la tecla se ha introducido, el editor colocará el texto \texttt{INSERTAR} o \texttt{INSERT} en la parte inferior de la pantalla, denoatando que el modo de inserción está activado. 
	\item \textit{VISUAL}: Este entorno se encarga de función de copiar y pegar contenido de las lineas. Para entrar a este modo, se debe introducir \texttt{v} con el teclado.
Una vez que el modo \textit{VISUAL} está activado debe aparecer en pantalla la leyenda \texttt{VISUAL} en la esquina inferior izquierda. 
\end{itemize}

\section{}
\end{document}

